% !TEX TS-program = pdflatex
% !TEX encoding = UTF-8 Unicode

% This file is a template using the "beamer" package to create slides for a talk or presentation
% - Giving a talk on some subject.
% - The talk is between 15min and 45min long.
% - Style is ornate.

% MODIFIED by Jonathan Kew, 2008-07-06
% The header comments and encoding in this file were modified for inclusion with TeXworks.
% The content is otherwise unchanged from the original distributed with the beamer package.

\documentclass{beamer}


% Copyright 2004 by Till Tantau <tantau@users.sourceforge.net>.
%
% In principle, this file can be redistributed and/or modified under
% the terms of the GNU Public License, version 2.
%
% However, this file is supposed to be a template to be modified
% for your own needs. For this reason, if you use this file as a
% template and not specifically distribute it as part of a another
% package/program, I grant the extra permission to freely copy and
% modify this file as you see fit and even to delete this copyright
% notice.


\mode<presentation>
{
\usetheme{Warsaw}
\usefonttheme[onlylarge]{structurebold}
\usecolortheme{seagull}
\setbeamerfont*{frametitle}{size=\normalsize,series=\bfseries}
\setbeamertemplate{navigation symbols}{\insertframenumber/\inserttotalframenumber}{}
% or ...

\setbeamercovered{transparent}
% or whatever (possibly just delete it)
}




\usepackage[english]{babel}
% or whatever
\usepackage{kotex}
\usepackage{tabularx}
\usepackage{colortbl}
\usepackage{multirow}
\usepackage{booktabs}
\usepackage[utf8]{inputenc}
\usepackage{import}

% or whatever

\usepackage{times}
\usepackage[T1]{fontenc}
%\setbeamertemplate{itemize items}[default]
\setbeamertemplate{itemize item}{$\bullet$}
\setbeamertemplate{itemize subitem}{$\circ$}
\setbeamertemplate{itemize subsubitem}{$\cdot$}

\setbeamertemplate{itemize/enumerate body begin}{\footnotesize}
\setbeamertemplate{itemize/enumerate subbody begin}{\scriptsize}
\setbeamertemplate{itemize/enumerate subsubbody begin}{\scriptsize}

% huge
% large
% normalsize
% small
% footnotesize
% scriptsize
% tiny


\setlength{\leftmargin}{5pt}
\setlength{\leftmargini}{5pt}
\setlength{\leftmarginii}{5pt}
\setlength{\leftmarginiii}{5pt}


% Or whatever. Note that the encoding and the font should match. If T1
% does not look nice, try deleting the line with the fontenc.

\title[WebOS 3.0 QA spec \alert{LGE CONFIDENTIAL}] % (optional, use only with long paper titles)
{LGSE QA Test Manual Ver.7}

\subtitle
{for H15/M16/K2L/A5LR/M2 - WebOS 3.0} % (optional)


\author[Bong-Jin Lee, Kichul Kim] % (optional, use only with lots of authors)
{Bong-Jin Lee (bongjin.lee@lge.com)\\Kichul Kim (kichul.kim@lge.com)}
% - Use the \inst{?} command only if the authors have different
%   affiliation.

\institute[IPT team, SIC lab., LG Electronics] % (optional, but mostly needed)
{
  IPT team, SIC lab., LG Electronics \\
  Release Link: (http://collab.lge.com/main/x/0hVwG)
  }
% - Use the \inst command only if there are several affiliations.
% - Keep it simple, no one is interested in your street address.

\date[Short Occasion] % (optional)
{\today\\ \alert{LGE CONFIDENTIAL}}

\subject{Talks}
% This is only inserted into the PDF information catalog. Can be left
% out.



% If you have a file called "university-logo-filename.xxx", where xxx
% is a graphic format that can be processed by latex or pdflatex,
% resp., then you can add a logo as follows:

% \pgfdeclareimage[height=0.5cm]{university-logo}{university-logo-filename}
% \logo{\pgfuseimage{university-logo}}



% Delete this, if you do not want the table of contents to pop up at
% the beginning of each subsection:
\AtBeginSubsection[]
{
  \begin{frame}<beamer>{Outline}
    \tableofcontents[currentsection,currentsubsection]
  \end{frame}
}


%%%%%%%%%%%%%%%%%%%%%%%%%%%%%%%%%%%%%%%%%%%%%%%%%%%%%%%%%%%%%%%%%%%%%%%%
\begin{document}
\begin{frame}
  \titlepage
\end{frame}


%%%%%%%%%%%%%%%%%%%%%%%%%%%%%%%%%%%%%%%%%%%%%%%%%%%%%%%%%%%%%%%%%%%%%%%%
\begin{frame}[t]{Test Spec. 사용 및 관리}

 \begin{itemize}
 \item Test Spec 변경의 원인은 아래와 같습니다.
 	\begin{itemize}
	\item 기능의 특성상 각 기능의 parameter 값이 튜닝 후 변경되면, 특성들이 변경됩니다.
	\item 이 때문에, parameter 변경을 하게 되면 Spec도 변경되게 됩니다.
	\end{itemize}
\end{itemize}

 \begin{itemize}
 \item Test Spec 사용
 	\begin{itemize}
 	\item Test Spec은 현재 기준으로 가장 최종 Spec을 적용하여 사용하시면 됩니다.
	\end{itemize}
\end{itemize}

 \begin{itemize}
 \item Test Spec 관리
 	\begin{itemize}
 	\item parameter 값을 튜닝 및 관리하는 TV 음팀과 SIC 연구소 IPT팀에서 Spec 배포 및 관리하며, 최종 버전 문서만 관리 대상입니다.
	\end{itemize}
\end{itemize}

\end{frame}

%%%%%%%%%%%%%%%%%%%%%%%%%%%%%%%%%%%%%%%%%%%%%%%%%%%%%%%%%%%%%%%%%%%%%%%%
\begin{frame}[t]{History}
\begin{itemize}
\begin{scriptsize}
\item 2015.12.29\_Ver.7 OLED 55/65C6 스펙 추가 / UH93, UH98, UH96, UH95, UH85, UH86, UH77, UH78, UH87, UH75 스펙 추가 / UH85 surround, clearvoice, sports, game 스펙 변경
\item 2015.12.3\_Ver.6 Smart Sound, Game, 3D Sound Zooming 스펙 변경 (5kHz 테스트 제거)
\item 2015.11.26\_Ver.5 Autovolume 스펙 변경
\item 2015.11.25\_Ver.4 LH66, LH65, LH60, LH58, LH57 스펙 추가
\item 2015.11.06\_Ver.3 UH86, UH85, UH78, UH77 스펙 추가
\item 2015.10.29\_Ver.2 UH66, UH65 스펙 변경 / UH68 스펙 추가
\item 2015.09.09\_Ver.1 UH66, UH65 스펙 추가
\end{scriptsize}
\end{itemize}
\end{frame}



%%%%%%%%%%%%%%%%%%%%%%%%%%%%%%%%%%%%%%%%%%%%%%%%%%%%%%%%%%%%%%%%%%%%%%%%
\begin{frame}{}
\tableofcontents
\huge Audio Precision 설정 방법\\
\end{frame}



%%%%%%%%%%%%%%%%%%%%%%%%%%%%%%%%%%%%%%%%%%%%%%%%%%%%%%%%%%%%%%%%%%%%%%%%
\begin{frame}[t]{Signal Path Setting}
\begin{itemize}
\item 스피커앰프의 출력을 8옴 1\% Dummy 저항을 거쳐서 Audio Precision에 입력으로 연결합니다.
\item 테스트 신호는 Audio Precision에서 재생하는 사운드로 HDMI 혹은 unbalaced (RCA)를 이용합니다.
\item Project -> Signal Path Setup -> Level을 활성화 합니다.
	\begin{itemize}
	\item Sampling rate: 48kHz, Bit depth: 16bits
	\item Waveform: sine, Frequency: 1kHz, Level: -12dBFS (25\%FS, 500mVrms)
	\item 특별한 언급이 없는 한, 채널 L/R은 모두 같은 크기, 같은 위상의 신호를 넣어야 합니다.
	\end{itemize}
\end{itemize}

\begin{figure}[r]
	\includegraphics[height=0.4\textwidth]{figure/apsetting/signalPath.png}
\end{figure}
\end{frame}



%%%%%%%%%%%%%%%%%%%%%%%%%%%%%%%%%%%%%%%%%%%%%%%%%%%%%%%%%%%%%%%%%%%%%%%%
\begin{frame}[t]{THD+N Check}
\begin{itemize}
\item 테스트 신호는 Audio Precision에서 재생하는 사운드로 HDMI 혹은 unbalaced (RCA)를 이용합니다.
\item Project -> Signal Path Setup -> THD+N Ratio를 활성화 합니다.
	\begin{itemize}
	\item Waveform: sine, Level: 0dBFS (100\%FS, 2Vrms)
	\item 스펙항목의 Frequency를 입력 합니다.
	\end{itemize}
\end{itemize}

\begin{figure}[b]
\includegraphics[height=0.4\textwidth]{figure/apsetting/thdn.png}
\end{figure}

\end{frame}


%%%%%%%%%%%%%%%%%%%%%%%%%%%%%%%%%%%%%%%%%%%%%%%%%%%%%%%%%%%%%%%%%%%%%%%%
\begin{frame}[t]{Interchannel Phase Check}
\begin{itemize}
\item 테스트 신호는 Audio Precision에서 재생하는 사운드로 HDMI 혹은 unbalaced (RCA)를 이용합니다.
\item Project -> Interchannel Phase를 활성화 합니다.
	\begin{itemize}
	\item Waveform: sine, Level: -12dBFS (25\%FS, 0.5Vrms)
	\item Ref Channel: Ch1, Meter Range: -180 -> 180 deg
	\item 스펙항목의 Frequency를 입력 합니다.
	\end{itemize}
\end{itemize}


\begin{figure}[b]
\includegraphics[height=0.4\textwidth]{figure/apsetting/interchannelPhase.png}
\end{figure}

\end{frame}

%%%%%%%%%%%%%%%%%%%%%%%%%%%%%%%%%%%%%%%%%%%%%%%%%%%%%%%%%%%%%%%%%%%%%%%%
\begin{frame}[t]{Frequency Response Check}
\begin{itemize}
\item 테스트 신호는 Audio Precision에서 재생하는 사운드로 HDMI 혹은 unbalaced (RCA)를 이용합니다.
\item Project -> Frequency Response -> Level을 활성화 합니다.
	\begin{itemize}
	\item Start Frequency: 20Hz, Stop Frequency: 20kHz
	\item Level: -12dBFS (25\%FS, 500mVrms)
	\item Pre-Sweep: 100ms, Sweep: 3s
	\end{itemize}
\item 기능이 off일 때를 기준 (ref)으로 하고, on일 때의 출력레벨 차이값을 측정합니다.
\end{itemize}

\begin{figure}[r]
\includegraphics[height=0.4\textwidth]{figure/apsetting/frequencyResponse.png}
\end{figure}

\end{frame}


%%%%%%%%%%%%%%%%%%%%%%%%%%%%%%%%%%%%%%%%%%%%%%%%%%%%%%%%%%%%%%%%%%%%%%%%
\begin{frame}[t]{Level Check}
\begin{itemize}
\item 테스트 신호는 Audio Precision에서 재생하는 사운드로 HDMI 혹은 unbalaced (RCA)를 이용합니다.
\item Project -> Signal Path Setup -> Level을 활성화 합니다.
	\begin{itemize}
	\item Waveform: sine, Level: -12dBFS (25\%FS, 500mVrms)
	\item 스펙항목의 Frequency를 입력 합니다.
	\end{itemize}
\item 기능이 off일 때를 기준 (ref)으로 하고, on일 때의 출력레벨 차이값을 측정합니다.
\end{itemize}

\begin{figure}[r]
\includegraphics[height=0.4\textwidth]{figure/apsetting/level.png}
\end{figure}

\end{frame}

%%%%%%%%%%%%%%%%%%%%%%%%%%%%%%%%%%%%%%%%%%%%%%%%%%%%%%%%%%%%%%%%%%%%%%%%
\begin{frame}[t]{Antiphase Level Check}
\begin{itemize}
\item 테스트 신호는 Audio Precision에서 재생하는 사운드로 HDMI 혹은 unbalaced (RCA)를 이용합니다.
\item Project -> Signal Path Setup -> Level을 활성화 합니다.
	\begin{itemize}
	\item Waveform: split phase, Level: -12dBFS (25\%FS, 500mVrms), Phase B: 180 deg
	\item 스펙항목의 Frequency를 입력 합니다.
	\end{itemize}
\item 기능이 off일 때를 기준 (ref)으로 하고, on일 때의 출력레벨 차이값을 측정합니다.
\end{itemize}

\begin{figure}[r]
\includegraphics[height=0.4\textwidth]{figure/apsetting/antiphaseLevel.png}
\end{figure}
\end{frame}


%%%%%%%%%%%%%%%%%%%%%%%%%%%%%%%%%%%%%%%%%%%%%%%%%%%%%%%%%%%%%%%%%%%%%%%%
\begin{frame}[t]{Autovolume Level Check}
\begin{itemize}
\item 테스트 신호는 Audio Precision에서 재생하는 사운드로 HDMI 혹은 USB 입력을 이용합니다.
\item Project -> Signal Path Setup -> Level을 활성화 합니다.
\item 테스트 입력
	\begin{itemize}
	\item Compensation 테스트
		\begin{itemize}
		\item USB 입력 (Autovolume.wav/mp3) 혹은 AP의 HDMI Source (sine, 1kHz, 16\%FS)
		\end{itemize}
	\item 1번 테스트
		\begin{itemize}
		\item USB 입력 (autovolume1.wav/mp3) 혹은 AP의 HDMI Source (sine, 1kHz, 1\%FS)
		\end{itemize}
	\item 2번 테스트
		\begin{itemize}
		\item USB 입력 (autovolume2.wav/mp3) 혹은 AP의 HDMI Source (sine, 1kHz, 90\%FS)
		\end{itemize}
	\end{itemize}
\item 기능이 off일 때를 기준 (ref)으로 하고, on일 때의 출력레벨 차이값을 측정합니다.
\item Compensation Test의 측정값을 Autovolume Test1, Autovolume Test2의 결과에 보상합니다.
\end{itemize}
\end{frame}


%%%%%%%%%%%%%%%%%%%%%%%%%%%%%%%%%%%%%%%%%%%%%%%%%%%%%%%%%%%%%%%%%%%%%%%%
\begin{frame}[t]{Impulse Response Check}
\begin{itemize}
\item 테스트 신호는 Audio Precision에서 재생하는 사운드로 HDMI 혹은 unbalaced (RCA)를 이용합니다.
\item Project -> Continuous Sweep -> Impulse Response를 활성화 합니다.
	\begin{itemize}
	\item Start Frequency: 20Hz, Stop Frequency: 20kHz
	\item Level: -12dBFS (25\%FS, 500mVrms)
	\item Pre-Sweep: 100ms, Sweep: 3s
	\end{itemize}
\end{itemize}

\begin{figure}[r]
\includegraphics[height=0.4\textwidth]{figure/apsetting/impulseResponse.png}
\end{figure}

\end{frame}


%%%%%%%%%%%%%%%%%%%%%%%%%%%%%%%%%%%%%%%%%%%%%%%%%%%%%%%%%%%%%%%%%%%%%%%%
\begin{frame}[t]{참고}
	\begin{itemize}
	\item 본 매뉴얼은 APx500 v3.4버전 프로그램을 기준으로 작성되었습니다.
	\end{itemize}

	\begin{figure}
		\begin{center}
		\includegraphics[width=0.7\textwidth]{figure/ap/about_apx500.jpg}
		\end{center}
	\end{figure}
\end{frame}


%%%%%%%%%%%%%%%%%%%%%%%%%%%%%%%%%%%%%%%%%%%%%%%%%%%%%%%%%%%%%%%%%%%%%%%%
%%%%%%%%%%%%%%%%%%%%%%%%%%%%%%%%%%%%%%%%%%%%%%%%%%%%%%%%%%%%%%%%%%%%%%%%
%%%%%%%%%%%%%%%%%%%%%%%%%%%%%%%%%%%%%%%%%%%%%%%%%%%%%%%%%%%%%%%%%%%%%%%%
%%%%%%%%%%%%%%%%%%%%%%%%%%%%%%%%%%%%%%%%%%%%%%%%%%%%%%%%%%%%%%%%%%%%%%%%



\setbeamertemplate{itemize/enumerate body begin}{\tiny}
\setbeamertemplate{itemize/enumerate subbody begin}{\tiny}
\setbeamertemplate{itemize/enumerate subsubbody begin}{\tiny}

%%%%%%%%%%%%%%%%%%%%%%%%%%%%%%%%%%%%%%%%%%%%%%%%%%%%%%%%%%%%%%%%%%%%%%%%
%%%%%%%%%%%%%%%%%%%%%%%%%%%%%%%%%%%%%%%%%%%%%%%%%%%%%%%%%%%%%%%%%%%%%%%%


%%%%%%%%%%%%%%%%%%%%%%%%%%%%%%%%%%%%%%%%%%%%%%%%%%%%%%%%%%%%%%%%%%%%%%%%
\begin{frame}[t]{}
\tableofcontents
\huge
\begin{itemize}
\large \item 다음 모델들의 스펙은 동일합니다.
\huge \item C6
\end{itemize}
\end{frame}

\subimport{../C6/}{C6.tex}


%%%%%%%%%%%%%%%%%%%%%%%%%%%%%%%%%%%%%%%%%%%%%%%%%%%%%%%%%%%%%%%%%%%%%%%% 
\begin{frame}[t]{}
\tableofcontents
\large
\begin{itemize}
\large \item 다음 모델들의 스펙은 동일합니다.
\large \item UH98
\large \item UH96
\large \item UH95
\large \item UH93
\large \item UH87
\large \item UH86
\large \item UH85
\large \item UH78
\large \item UH77
\large \item UH75
\end{itemize}
\end{frame}

\subimport{../UH93_UH98_UH96_UH95_UH85_UH86_UH77_UH78_UH87_UH75/}{UH77.tex}


%%%%%%%%%%%%%%%%%%%%%%%%%%%%%%%%%%%%%%%%%%%%%%%%%%%%%%%%%%%%%%%%%%%%%%%% 
\begin{frame}[t]{}
\tableofcontents
\huge
\begin{itemize}
\large \item 다음 모델들의 스펙은 동일합니다.
\huge \item UH68
\huge \item UH66
\huge \item UH65
\huge \item UH63
\huge \item UH61
\end{itemize}
\end{frame}

\subimport{../UH68_UH66_UH65_UH63_UH61/}{UH68.tex}





%%%%%%%%%%%%%%%%%%%%%%%%%%%%%%%%%%%%%%%%%%%%%%%%%%%%%%%%%%%%%%%%%%%%%%%% 
\begin{frame}[t]{}
\tableofcontents
\huge
\begin{itemize}
\large \item 다음 모델들의 스펙은 동일합니다.
\huge \item LH66
\huge \item LH65
\huge \item LH60
\huge \item LH58
\huge \item LH57
\end{itemize}
\end{frame}

\subimport{../LH66_LH65_LH60_LH58_LH57/}{LH66.tex}



\end{document}
